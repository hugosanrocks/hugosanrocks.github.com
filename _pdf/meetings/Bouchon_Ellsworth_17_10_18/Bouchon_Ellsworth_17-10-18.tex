%-_-_-_-_-_-_-_-_-_-_-_-_-_-_-_-_-_-_-_-_-_-_-_-_-_-_-_-_-_-_-_-_-_-_-_
% TUNING OF THE THEME AND PACKAGES

%Here should be chosen the global font size and aspect ratio
%\documentclass[aspectratio=169,9pt]{beamer}
\documentclass[aspectratio=43,9pt]{beamer}

%here is the advised tuning of the metroseiscope theme -> a seiscope tuned version of metropolis
\usetheme[sectionpage=seiscope, numbering=counter, progressbar=frametitle, background=light]{metroseiscope}
%possible choices are
%numbering=none,counter,fraction
%progressbar=foot,head,frametitle
\usepackage{pifont}
\usepackage{marvosym}
\usepackage{appendixnumberbeamer}
\usepackage[utf8x]{inputenc}
\usepackage[english]{babel}
\usepackage{hyperref}
\usepackage{natbib}
\usepackage{algorithm2e}
\usepackage{algorithmic}
\usepackage{amsmath}
\usepackage{amsfonts}
\usepackage{verbatim}
\usepackage{graphicx}
\usepackage{epsfig}
\usepackage{multicol}
\usepackage{float}
\usepackage{pdfpages}
\usepackage{setspace}
\usepackage{tikz}
\usepackage{amssymb}
\usetikzlibrary{shapes.geometric, arrows}
\usetikzlibrary{shapes.misc}

\usepackage{animate,media9} %,movie15}

\usepackage{color}
\input{Figs/rgb}




\tikzset{cross/.style={cross out, draw=black, minimum size=2*(#1-\pgflinewidth), inner sep=0pt, outer sep=0pt},
%default radius will be 1pt. 
cross/.default={20pt}}


\usepackage{datetime}
%\usepackage{unicode-math} %work with Xelatex only

\usepackage{animate}
\usepackage{perpage}       %reinit the footnote number at each page/slide
\MakePerPage{footnote}

%font of the seiscope logo used for the beamer also
%\setsansfont{Montserrat}


%correct rendering of some png, pdf figure, which appears weird on Adobe Reader
%\pdfpageattr{/Group <</S /Transparency /I true /CS /DeviceRGB>>}

%*********************
%footnote management
%--------------------
\let\oldfootnote\footnote
\renewcommand\footnote[1][]{\oldfootnote[frame,#1]}
%put defaultfootnote without number
%\renewcommand{\thefootnote}{\*}
%or manage it on the fly with
%\renewcommand{\thefootnote}{\arabic{footnote}}
%\renewcommand{\thefootnote}{\*}
%*********************

%size of the text, if required
%\setbeamerfont{footnote}{size=\scriptsize}
%\setbeamerfont{caption}{size=\scriptsize}


%path to logo and biblio -> to be adapted to your local directories management
\newcommand\dirlogo{/home/hugo/svn/SEISCOPE_ARTICLES/SEISCOPE/}
\newcommand\dirbiblio{/home/hugo/svn/SEISCOPE_ARTICLES/BIBLIO/}


% automatic logo in the header on slides -> should not be changed
\addtobeamertemplate{frametitle}{}{%
\begin{tikzpicture}[remember picture,overlay]
  \node[anchor=north east,yshift=0.5ex] at (current page.north east) {\includegraphics[height=3.3ex]{\dirlogo/seiscope_color_dark_background}};
  %\node[anchor=north east,yshift=0.5ex] at (current page.north east) {\includegraphics[height=3.3ex]{\dirlogo/seiscope_color_light_background}};
\end{tikzpicture}}



% END OF TUNING OF THE THEME AND PACKAGES
%-_-_-_-_-_-_-_-_-_-_-_-_-_-_-_-_-_-_-_-_-_-_-_-_-_-_-_-_-_-_-_-_-_-_-_


%-_-_-_-_-_-_-_-_-_-_-_-_-_-_-_-_-_-_-_-_-_-_-_-_-_-_-_-_-_-_-_-_-_-_-_
% INFORMATIONS
\title{Articles: Bouchon et al. (2011) and Ellsworth et al. (2018)}								% TITLE
\subtitle{Pre-slip or cascading earthquakes model for M$_w$7.6 1999 Izmit earthquake?}					% SUBTITLE
\setbeamertemplate{frame footer}{Earthquake precursors - Reading group}		% BOTTOM TITLE ON EACH SLIDES
\newdate{date}{05}{01}{2017}							% DEFINE DATE
\date{\today}
\author{Hugo S. S\'anchez-Reyes}								% AUTHOR
\institute{ \begin{center} \includegraphics[width=0.55\linewidth]{Figs/fig_izmit.jpg}  \end{center} }							% INSTITUTE
\titlegraphic{\hfill \includegraphics[height=0.8cm]{Figs/logo-is.png} \, \includegraphics[height=0.8cm]{Figs/logo-uga.jpg} \vskip 4cm}	% Edit path to seiscope_color_light_background
\date{ ISTerre, Universit\'e Grenoble Alpes}
%-_-_-_-_-_-_-_-_-_-_-_-_-_-_-_-_-_-_-_-_-_-_-_-_-_-_-_-_-_-_-_-_-_-_-_




%-_-_-_-_-_-_-_-_-_-_-_-_-_-_-_-_-_-_-_-_-_-_-_-_-_-_-_-_-_-_-_-_-_-_-_
% BEGIN DOCUMENT
%-_-_-_-_-_-_-_-_-_-_-_-_-_-_-_-_-_-_-_-_-_-_-_-_-_-_-_-_-_-_-_-_-_-_-_
\begin{document}

% TITLE PAGE
\maketitle



\section{Bouchon et al. (2011): Pre-slip observations before the M$_w$7.6 1999 Izmit earthquake.}

\begin{frame}{Waveform similirity: coincidence or same source?}
 
 \begin{minipage}{0.45\linewidth}
 All of the well-distinguished events share nearly the same waveforms:
 \begin{itemize}
  \item  7 foreshocks well identified with a receiver $<$14 km epicenter offset.
  \item $\approx$ 40 events were found using cross-correlation with a template.
 \end{itemize}

 \end{minipage} \,
 \begin{minipage}{0.5\linewidth}
  \includegraphics[width=1\linewidth]{Figs/fig1.jpg}
 \end{minipage}

 
 
\end{frame}


\begin{frame}{18 foreshocks visually identified}
   \vskip -0.3cm
   Surprising features: \\
   \vskip 0.2cm
 \begin{minipage}{0.45\linewidth}
   \begin{itemize}
    \item S-minus-P traveltime of 2.4 (s).
    \item Some of them separated by 5 (s) between them.
   \end{itemize}
 \end{minipage}
 \begin{minipage}{0.45\linewidth}
   \begin{itemize}
    \item Magnitudes ranging from $[0.3, 2.7]$.
    \item Very similar waveforms.
   \end{itemize}
 \end{minipage} \\
 \begin{center}
  \includegraphics[width=0.7\linewidth]{Figs/fig2.jpg}  
 \end{center}

\end{frame}


\begin{frame}{Other similarities?}

 \begin{minipage}{0.52\linewidth}
  \includegraphics[width=1\linewidth]{Figs/fig4.jpg} 
 \end{minipage} \,
\begin{minipage}{0.45\linewidth}
  \includegraphics[width=1\linewidth]{Figs/fig4_1-2.jpg} \\
 Cross-correlating the 1st and 2nd foreshocks:
 \begin{itemize}
  \item  S-minus-P travel times differ $\approx$ 0.0006 s.
  \item  $\approx$ 5 m distance from one each other.
 \end{itemize}
\end{minipage}
\vskip 0.7cm
\centering The distance of $\approx$ 5 m is not between the sources but the 
projection of the source distance projected to the ray path.
 
\end{frame}



\begin{frame}{Why they are so similar?}

\vskip 0.2cm
Applying the same cross-correlation analysis
for all the possible P and S couples: \\
\vskip 0.5cm
Any of the events differs in S-minus-P travel 
time by less than 0.0024 s from the majority of the other events
\vskip 0.5cm

\centering {\bf This implies that any one shock is
located within 20 m or less from the majority of
the other events.} \\
\vskip 0.2cm
\includegraphics[width=0.3\linewidth]{Figs/fig_dist.jpg}

{\bf $\longrightarrow$ All of the events originate from an area of the fault that is 
no larger than the size of the largest events.}


\end{frame}

\begin{frame}{Activity accelerated}

How?
\begin{itemize}
 \item $4^{th}$ largest event occurred 43 min before the mainshock
 \item $3^{th}$ largest event occurred 20 min before the mainshock
 \item $2^{nd}$ largest event occurred 12 min before the mainshock
 \item $1^{st}$ largest event occurred 1 min 45 s before the mainshock
\end{itemize}
\vskip 0.2cm
Acceleration increase again 1 min before the earthquake:
\begin{itemize}
 \item One shock occurred 0.14 s before the mainshock (magnitude $\approx$2.0).
 \item Another shocked 0.07 s later (P-pulse bigger than previous ones).
\end{itemize}
 
\end{frame}


\begin{frame}{Questions related to the acceleration}
 
 Magnitude is not increasing in that critical one minute before?? \\
 but in the others yes??

 why we can not see this phenomena for other earthquakes?? \\
 there are receivers closer than 14 km nowadays! 

\end{frame}


\begin{frame}{Why do they look so similar:}

\begin{minipage}{0.45\linewidth}

 Their spectral corner frequencies is higher than
 the maximum frequency displayed (35 Hz): \\
 {\bf (seen as false point sources)} \\
 \vskip 0.1cm
 Or, the two events may have nearly the same corner frequencies, 
 because spectral amplitude drops off rapidly beyond an event corner frequency. \\
 \vskip 0.1cm
 Same patch, same length ($\approx$ 300 m) \\
 but different slip: \\
 1st a little less than 1 cm (0.8 cm) \\ 
 2nd a little less than 1 mm (0.8 mm) \\ 
 1st stress drop of ~2.6 MPa  \\
 2nd stress drop of ~0.3 MPa 
%The spectra you showed is from the S wave.... this arrivval does 
%not represent the source length (from my experience). The source 
%length might not be related to this peak/amplitude of the S wave
\end{minipage} \,
\begin{minipage}{0.45\linewidth}
 \includegraphics[width=1\linewidth]{Figs/fig5.jpg} \\
 {\small (A to C) Comparison of the S-wave ground-velocity spectra of some events. The peak recorded amplitude of
 each event is given in parenthesis and is expressed in micrometers per second.}
\end{minipage}

 
\end{frame}

\begin{frame}{Creep loading surrounding areas?}

 {\bf Repetive earthquake due to creep:} \\
 \vskip 0.2cm
 They usually have recurrent times of months or years
 and similar magnitudes. \\
 \vskip 0.5cm

 {\bf In this case, separated by minutes!} \\
 \vskip 0.2cm
 A possible reason for the magnitude difference 
 observed here is that the interevent time of the 
 pre-Izmit shocks is extremely short, possibly forcing 
 the response of the patch.

  \vskip 0.4cm
\centering {\bf patch responds not only to the loading but also to
the loading rate, which may be highly irregular.}

 
\end{frame}

\begin{frame}{Creep signature in the background noise?}
 \vskip -0.4cm
 After first the foreshock, the seismic gound motion content 
 at low-frequencies increased significantly.  {\bf Its energy is at least below 2Hz.}
 \begin{center}
 \vskip -0.3cm
   \includegraphics[width=0.7\linewidth]{Figs/fig6.jpg}
 \end{center}
 \vskip -0.3cm
 No location can be provided for this increase in the low-frequency content.
 
 \begin{center}
  {\bf Hypothesis: Maybe it is the signature of the creep
  happening around the hypocentral area.}  
 \end{center}

\end{frame}



\begin{frame}{Summary}
 
 These observations show that this particular earthquake 
 was preceded by a phase of slow slip occurring at the 
 base of the brittle crust. \\
 \vskip 0.2cm

 This observations (long duration nucleation phase)
 and similarity between foreshocks and mainshock are encouraging
 for early-warning systems... \\
 \vskip 0.2cm
 but
 \vskip 0.2cm
 other well-recorded earthquakes, 1999 Chi-Chi (Taiwan) or 
 2004 Parkfield (California), o not show evidence for similar 
 foreshocks or nucleation events.
 \vfill
\end{frame}


\begin{frame}{Summary}
 
 \includegraphics{Figs/fig_1ells.jpg}
 
\end{frame}



\begin{frame}{Summary}
 
 \includegraphics{Figs/fig_2ells.jpg}
 
\end{frame}



\begin{frame}{Summary}
 
 \includegraphics{Figs/fig_3ells.jpg}
 
\end{frame}



\begin{frame}{Summary}
 
 \includegraphics{Figs/fig_4ells.jpg}
 
\end{frame}


\end{document}


