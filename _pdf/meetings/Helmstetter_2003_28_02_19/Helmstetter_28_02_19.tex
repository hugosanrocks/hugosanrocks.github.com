%-_-_-_-_-_-_-_-_-_-_-_-_-_-_-_-_-_-_-_-_-_-_-_-_-_-_-_-_-_-_-_-_-_-_-_
% TUNING OF THE THEME AND PACKAGES

%Here should be chosen the global font size and aspect ratio
%\documentclass[aspectratio=169,9pt]{beamer}
\documentclass[aspectratio=43,9pt]{beamer}

%here is the advised tuning of the metroseiscope theme -> a seiscope tuned version of metropolis
\usetheme[sectionpage=seiscope, numbering=counter, progressbar=frametitle, background=light]{metroseiscope}
%possible choices are
%numbering=none,counter,fraction
%progressbar=foot,head,frametitle
\usepackage{pifont}
\usepackage{marvosym}
\usepackage{appendixnumberbeamer}
\usepackage[utf8x]{inputenc}
\usepackage[english]{babel}
\usepackage{hyperref}
\usepackage{natbib}
\usepackage{algorithm2e}
\usepackage{algorithmic}
\usepackage{amsmath}
\usepackage{amsfonts}
\usepackage{verbatim}
\usepackage{graphicx}
\usepackage{epsfig}
\usepackage{multicol}
\usepackage{float}
\usepackage{pdfpages}
\usepackage{setspace}
\usepackage{tikz}
\usepackage{amssymb}
\usetikzlibrary{shapes.geometric, arrows}
\usetikzlibrary{shapes.misc}

\usepackage{animate,media9} %,movie15}

\usepackage{color}
\input{Figs/rgb}




\tikzset{cross/.style={cross out, draw=black, minimum size=2*(#1-\pgflinewidth), inner sep=0pt, outer sep=0pt},
%default radius will be 1pt. 
cross/.default={20pt}}


\usepackage{datetime}
%\usepackage{unicode-math} %work with Xelatex only

\usepackage{animate}
\usepackage{perpage}       %reinit the footnote number at each page/slide
\MakePerPage{footnote}

%font of the seiscope logo used for the beamer also
%\setsansfont{Montserrat}


%correct rendering of some png, pdf figure, which appears weird on Adobe Reader
%\pdfpageattr{/Group <</S /Transparency /I true /CS /DeviceRGB>>}

%*********************
%footnote management
%--------------------
\let\oldfootnote\footnote
\renewcommand\footnote[1][]{\oldfootnote[frame,#1]}
%put defaultfootnote without number
%\renewcommand{\thefootnote}{\*}
%or manage it on the fly with
%\renewcommand{\thefootnote}{\arabic{footnote}}
%\renewcommand{\thefootnote}{\*}
%*********************

%size of the text, if required
%\setbeamerfont{footnote}{size=\scriptsize}
%\setbeamerfont{caption}{size=\scriptsize}


%path to logo and biblio -> to be adapted to your local directories management
\newcommand\dirlogo{/home/hugo/svn/SEISCOPE_ARTICLES/SEISCOPE/}
\newcommand\dirbiblio{/home/hugo/svn/SEISCOPE_ARTICLES/BIBLIO/}


% automatic logo in the header on slides -> should not be changed
\addtobeamertemplate{frametitle}{}{%
\begin{tikzpicture}[remember picture,overlay]
  \node[anchor=north east,yshift=0.5ex] at (current page.north east) {\includegraphics[height=3.3ex]{Figs/logo-is2}};
  %\node[anchor=north east,yshift=0.5ex] at (current page.north east) {\includegraphics[height=3.3ex]{\dirlogo/seiscope_color_light_background}};
\end{tikzpicture}}



% END OF TUNING OF THE THEME AND PACKAGES
%-_-_-_-_-_-_-_-_-_-_-_-_-_-_-_-_-_-_-_-_-_-_-_-_-_-_-_-_-_-_-_-_-_-_-_


%-_-_-_-_-_-_-_-_-_-_-_-_-_-_-_-_-_-_-_-_-_-_-_-_-_-_-_-_-_-_-_-_-_-_-_
% INFORMATIONS
\title{Helmstetter et al. (2003)}								% TITLE
\subtitle{Mainshocks are aftershocks of conditional foreshocks:
How do foreshock statistical properties emerge from
aftershock laws}					% SUBTITLE
\setbeamertemplate{frame footer}{Earthquake precursors - Reading group}		% BOTTOM TITLE ON EACH SLIDES
\newdate{date}{05}{01}{2017}							% DEFINE DATE
\date{\today}
\author{Hugo S. S\'anchez-Reyes}								% AUTHOR
\institute{ \begin{center}  \end{center} }							% INSTITUTE
\titlegraphic{\hfill \includegraphics[height=0.8cm]{Figs/logo-is.png} \, \includegraphics[height=0.8cm]{Figs/logo-uga.jpg} \vskip 4cm}	% Edit path to seiscope_color_light_background
\date{ ISTerre, Universit\'e Grenoble Alpes}
%-_-_-_-_-_-_-_-_-_-_-_-_-_-_-_-_-_-_-_-_-_-_-_-_-_-_-_-_-_-_-_-_-_-_-_




%-_-_-_-_-_-_-_-_-_-_-_-_-_-_-_-_-_-_-_-_-_-_-_-_-_-_-_-_-_-_-_-_-_-_-_
% BEGIN DOCUMENT
%-_-_-_-_-_-_-_-_-_-_-_-_-_-_-_-_-_-_-_-_-_-_-_-_-_-_-_-_-_-_-_-_-_-_-_
\begin{document}

% TITLE PAGE
\maketitle


\section{Introduction: Pre-slip vs cascade of foreshocks}


\begin{frame}{Two points of view}
 \begin{minipage}{0.45\linewidth}
 \begin{center}
   {\large Pre-slip model}
 \end{center}
   Foreshocks are triggered by aseismic slip over an area surrounding
   the mainshock hypocentre. \\
   \vskip 0.5cm
   {\large Consequences:} \\
   \vskip 0.1cm
   The underlying aseismic slip might be a precursor to 
   the earthquake.
 \end{minipage} \quad
 \begin{minipage}{0.45\linewidth}
 \begin{center}
  {\large Cascade model}
 \end{center}
  Foreshocks might occur by neighbour-to-neighbour stress transfer between them 
  without an aseismic slip component. \\
  \vskip 0.3cm
  {\large Consequences:} \\
  \vskip 0.1cm
  The foreshocks are no different than any other set of clustered earthquakes.\\
  \vskip 0.1cm
  The mainshock is just a random outcome of triggering.
 \end{minipage}
 
\end{frame}


\begin{frame}
 {}
 
 \begin{minipage}{0.48\linewidth}
   \includegraphics[width=1\linewidth]{Figs/fig1}
 \end{minipage}
 \begin{minipage}{0.48\linewidth}

 \end{minipage}

 
\end{frame}


\begin{frame}
 {}
 
 \begin{minipage}{0.48\linewidth}
   \includegraphics[width=1\linewidth]{Figs/fig2}
 \end{minipage}
 \begin{minipage}{0.48\linewidth}

 \end{minipage}

 
\end{frame}


\begin{frame}
 {}
 
 \begin{minipage}{0.48\linewidth}
   \includegraphics[width=1\linewidth]{Figs/fig3}
 \end{minipage}
 \begin{minipage}{0.48\linewidth}

 \end{minipage}

 
\end{frame}


\begin{frame}
 {}
 
 \begin{minipage}{0.48\linewidth}
   \includegraphics[width=1\linewidth]{Figs/fig4}
 \end{minipage}
 \begin{minipage}{0.48\linewidth}

 \end{minipage}

 
\end{frame}



\begin{frame}
 {}
 
 \begin{minipage}{0.48\linewidth}
   \includegraphics[width=1\linewidth]{Figs/fig5}
 \end{minipage}
 \begin{minipage}{0.48\linewidth}

 \end{minipage}

 
\end{frame}



\begin{frame}
 {}
 
 \begin{minipage}{0.48\linewidth}
   \includegraphics[width=1\linewidth]{Figs/fig6}
 \end{minipage}
 \begin{minipage}{0.48\linewidth}

 \end{minipage}

 
\end{frame}



\begin{frame}
 {}
 
 \begin{minipage}{0.48\linewidth}
   \includegraphics[width=1\linewidth]{Figs/fig7}
 \end{minipage}
 \begin{minipage}{0.48\linewidth}

 \end{minipage}

 
\end{frame}



\begin{frame}
 {}
 
 \begin{minipage}{0.48\linewidth}
   \includegraphics[width=1\linewidth]{Figs/fig8}
 \end{minipage}
 \begin{minipage}{0.48\linewidth}

 \end{minipage}

 
\end{frame}




\begin{frame}
 {}
 
 \begin{minipage}{0.48\linewidth}
   \includegraphics[width=1\linewidth]{Figs/fig9}
 \end{minipage}
 \begin{minipage}{0.48\linewidth}

 \end{minipage}

 
\end{frame}




\begin{frame}
 {}
 
 \begin{minipage}{0.48\linewidth}
   \includegraphics[width=1\linewidth]{Figs/fig10}
 \end{minipage}
 \begin{minipage}{0.48\linewidth}

 \end{minipage}

 
\end{frame}



\end{document}


