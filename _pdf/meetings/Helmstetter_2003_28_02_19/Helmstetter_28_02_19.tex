%-_-_-_-_-_-_-_-_-_-_-_-_-_-_-_-_-_-_-_-_-_-_-_-_-_-_-_-_-_-_-_-_-_-_-_
% TUNING OF THE THEME AND PACKAGES

%Here should be chosen the global font size and aspect ratio
%\documentclass[aspectratio=169,9pt]{beamer}
\documentclass[aspectratio=43,9pt]{beamer}

%here is the advised tuning of the metroseiscope theme -> a seiscope tuned version of metropolis
\usetheme[sectionpage=seiscope, numbering=counter, progressbar=frametitle, background=light]{metroseiscope}
%possible choices are
%numbering=none,counter,fraction
%progressbar=foot,head,frametitle
\usepackage{pifont}
\usepackage{marvosym}
\usepackage{appendixnumberbeamer}
\usepackage[utf8x]{inputenc}
\usepackage[english]{babel}
\usepackage{hyperref}
\usepackage{natbib}
\usepackage{algorithm2e}
\usepackage{algorithmic}
\usepackage{amsmath}
\usepackage{amsfonts}
\usepackage{verbatim}
\usepackage{graphicx}
\usepackage{epsfig}
\usepackage{multicol}
\usepackage{float}
\usepackage{pdfpages}
\usepackage{setspace}
\usepackage{tikz}
\usepackage{amssymb}
\usetikzlibrary{shapes.geometric, arrows}
\usetikzlibrary{shapes.misc}

\usepackage{animate,media9} %,movie15}

\usepackage{color}
\input{Figs/rgb}




\tikzset{cross/.style={cross out, draw=black, minimum size=2*(#1-\pgflinewidth), inner sep=0pt, outer sep=0pt},
%default radius will be 1pt. 
cross/.default={20pt}}


\usepackage{datetime}
%\usepackage{unicode-math} %work with Xelatex only

\usepackage{animate}
\usepackage{perpage}       %reinit the footnote number at each page/slide
\MakePerPage{footnote}

%font of the seiscope logo used for the beamer also
%\setsansfont{Montserrat}


%correct rendering of some png, pdf figure, which appears weird on Adobe Reader
%\pdfpageattr{/Group <</S /Transparency /I true /CS /DeviceRGB>>}

%*********************
%footnote management
%--------------------
\let\oldfootnote\footnote
\renewcommand\footnote[1][]{\oldfootnote[frame,#1]}
%put defaultfootnote without number
%\renewcommand{\thefootnote}{\*}
%or manage it on the fly with
%\renewcommand{\thefootnote}{\arabic{footnote}}
%\renewcommand{\thefootnote}{\*}
%*********************

%size of the text, if required
%\setbeamerfont{footnote}{size=\scriptsize}
%\setbeamerfont{caption}{size=\scriptsize}


%path to logo and biblio -> to be adapted to your local directories management
\newcommand\dirlogo{/home/hugo/svn/SEISCOPE_ARTICLES/SEISCOPE/}
\newcommand\dirbiblio{/home/hugo/svn/SEISCOPE_ARTICLES/BIBLIO/}


% automatic logo in the header on slides -> should not be changed
\addtobeamertemplate{frametitle}{}{%
\begin{tikzpicture}[remember picture,overlay]
  \node[anchor=north east,yshift=0.5ex] at (current page.north east) {\includegraphics[height=3.3ex]{Figs/logo-is2}};
  %\node[anchor=north east,yshift=0.5ex] at (current page.north east) {\includegraphics[height=3.3ex]{\dirlogo/seiscope_color_light_background}};
\end{tikzpicture}}



% END OF TUNING OF THE THEME AND PACKAGES
%-_-_-_-_-_-_-_-_-_-_-_-_-_-_-_-_-_-_-_-_-_-_-_-_-_-_-_-_-_-_-_-_-_-_-_


%-_-_-_-_-_-_-_-_-_-_-_-_-_-_-_-_-_-_-_-_-_-_-_-_-_-_-_-_-_-_-_-_-_-_-_
% INFORMATIONS
\title{Helmstetter et al. (2003)}								% TITLE
\subtitle{Mainshocks are aftershocks of conditional foreshocks:
How do foreshock statistical properties emerge from
aftershock laws}					% SUBTITLE
\setbeamertemplate{frame footer}{Earthquake precursors - Reading group}		% BOTTOM TITLE ON EACH SLIDES
\newdate{date}{05}{01}{2017}							% DEFINE DATE
\date{\today}
\author{Hugo S. S\'anchez-Reyes}								% AUTHOR
\institute{ \begin{center}  \end{center} }							% INSTITUTE
\titlegraphic{\hfill \includegraphics[height=0.8cm]{Figs/logo-is.png} \, \includegraphics[height=0.8cm]{Figs/logo-uga.jpg} \vskip 4cm}	% Edit path to seiscope_color_light_background
\date{ ISTerre, Universit\'e Grenoble Alpes}
%-_-_-_-_-_-_-_-_-_-_-_-_-_-_-_-_-_-_-_-_-_-_-_-_-_-_-_-_-_-_-_-_-_-_-_




%-_-_-_-_-_-_-_-_-_-_-_-_-_-_-_-_-_-_-_-_-_-_-_-_-_-_-_-_-_-_-_-_-_-_-_
% BEGIN DOCUMENT
%-_-_-_-_-_-_-_-_-_-_-_-_-_-_-_-_-_-_-_-_-_-_-_-_-_-_-_-_-_-_-_-_-_-_-_
\begin{document}

% TITLE PAGE
\maketitle


\section{Goal: }

\begin{frame}{What they wanted to answer:}

Is it possible to derive most if not all of the observed
phenomenology of foreshocks from the knowledge of only
the most basic and robust facts of earthquake phenomenology, 
namely the GR and Omori laws? \\
\vskip 0.2cm
They use what is maybe the simplest statistical model
of seismicity, the so-called ETAS model (epidemic-type aftershock sequence),
based only on the GR and Omori laws.\\
\vskip 0.2cm
1. We shall call ‘‘foreshock’’ of type I any event of
magnitude smaller than or equal to the magnitude of the
following event, then identified as a ‘‘mainshock.’’\\
\vskip 0.2cm
2. We shall also consider ‘‘foreshock’’ of type II, as any
earthquake preceding a large earthquake, defined as the
mainshock, independently of the relative magnitude of the
foreshock compared to that of the mainshock.\\
\vskip 0.2cm 
\begin{minipage}{0.45\linewidth}
 Inverse Omori law
 The rate 0 of earthquakes prior to a mainshock increases on average as a power law 
  $ \frac{1}{(t_c-t)^{p'}}$
\end{minipage} \quad
\begin{minipage}{0.45\linewidth}
 Here, we show that this law results from the
 direct Omori law for aftershocks describing the power law decay $\sim \frac{1}{(t-t_c)^p}$
\end{minipage}


\end{frame}




\begin{frame}
 {Aftershocks of the aftershocks}
 
 \begin{minipage}{0.48\linewidth}
   \includegraphics[width=1\linewidth]{Figs/fig1}
 \end{minipage}
 \begin{minipage}{0.48\linewidth}
  One realization of the ETAS model: \\
  - observed seismicity rate $\kappa (t)$ (noisy solid line) \\ 
  - average renormalized propagator $K(t)$ (solid line) \\
  - local propagator $\Phi_E (t)$ (dashed line) \\ \\
  The global aftershock rate $\kappa (t)$ is significantly higher
than the direct (or first generation) aftershock rate,
described by the local propagator $\Phi_E (t)$. \\ \\
Large fluctuations of
the seismicity rate correspond to the occurrence of large
aftershocks, which trigger their own aftershock sequence.
 \end{minipage}

 
\end{frame}


\begin{frame}
 {Appears the inverse Omori law!}
 
 %\begin{minipage}{0.48\linewidth}
 \begin{center}
   \vspace{-20pt}
   \includegraphics[width=0.8\linewidth]{Figs/fig2} \\  
 \end{center}
 %\end{minipage}
 %\begin{minipage}{0.48\linewidth}
  Foreshock (a) and aftershock (b) sequences generated by the ETAS model for $M=5.5$.
  The solid black line represents the mean (250seq) seismicity rate before and after a main 
  shock of magnitude $M = 5.5$. Compare to the direct Omori law, clearly observed after any 
  large mainshock, there are large fluctuations from one foreshock sequence to another. 
  The inverse Omori law (accelerating seismicity) is only observed when averaging over
  a large number of foreshock sequences.
 %\end{minipage}

 
\end{frame}


\begin{frame}
 {Type II aftershocks are independent of the mainshock energy.}
 
 %\begin{minipage}{0.48\linewidth}
 \begin{center}
   \vspace{-20pt}
   \includegraphics[width=0.6\linewidth]{Figs/fig3} \\
\end{center}
 %\end{minipage}
 %\begin{minipage}{0.48\linewidth}
 \vspace{-10pt}Direct and inverse Omori law for a numerical simulation showing the two 
 exponents $p = 1 -\theta$ for aftershocks and $p'=  1 - 2\theta$ for foreshocks
 of type II. 
 The rate of aftershocks (crosses) and foreshocks (circles) per main shock, averaged 
 over a large number of sequences, is shown as a function of the time $|t_c - t|$ 
 to the mainshock for different values of the mainshock magnitude [1.5,5].
 The number of aftershocks increases with the mainshock energy as $N \simeq E^a$ , whereas the
 number of foreshocks of type II is independent of the main shock energy.
 %\end{minipage}

 
\end{frame}


\begin{frame}
 {Type II aftershocks are independent of the mainshock energy.}
 
 %\begin{minipage}{0.48\linewidth}
 \begin{center}
     \includegraphics[width=0.6\linewidth]{Figs/fig4} \\
 \end{center}
 %\end{minipage}
 %\begin{minipage}{0.48\linewidth}
 Same as previous, but for $a = 0.8 \beta$, showing the larger relative ratio of foreshocks to
 aftershocks compared to the case a = 0.5 b.
 %\end{minipage}

 
\end{frame}



\begin{frame}
 {When the rate of aftershocks type I and II are close}
 
 %\begin{minipage}{0.48\linewidth}
 \begin{center}
   \includegraphics[width=0.5\linewidth]{Figs/fig5}  
 \end{center}
   Foreshock seismicity rate per mainshock for foreshocks of type II (circles) and foreshocks of
   type I (crosses, magnitudes from 3 to 6 of mainshocks) for a numerical simulations.
   The rate of foreshocks of type II is independent on the mainshock magnitude, while 
   the rate of foreshocks of type I increases with M. For large main shock magnitudes,
   the rate of foreshocks of type I is very close to that of foreshocks of type II.
 %\end{minipage}
 %\begin{minipage}{0.48\linewidth}

% \end{minipage}

 
\end{frame}



\begin{frame}
 {$p$ and $p'$ well-predicted when balanced the coupling between the earthquake energies 
and the seismic rate}
 
 %\begin{minipage}{0.48\linewidth}
   \includegraphics[width=0.8\linewidth]{Figs/fig6}
 %\end{minipage}
 %\begin{minipage}{0.48\linewidth}
  Exponents $p$ and $p'$ of the inverse and direct Omori laws obtained from numerical simulations
  of the ETAS model. The estimated values of $p$ (circles) for foreshocks and $'$ (crosses) 
  for aftershocks are shown as a function of $\theta$ in the case $\alpha = 0.5$ (a) and 
  as a function of $\alpha/\beta$ in the case $\theta = 0.2$ (b). The theoretical values of 
  $p$ are represented with dashed lines, and the theoretical prediction for $p$ is shown 
  as solid lines.
 %\end{minipage}

 
\end{frame}



\begin{frame}
 {Magnitude rate as the sum of two power laws!}
 
 %\begin{minipage}{0.48\linewidth}
   \includegraphics[width=0.8\linewidth]{Figs/fig7} \\
 %\end{minipage}
 %\begin{minipage}{0.48\linewidth}
  Magnitude distribution of foreshocks for two time periods, $|t_c - t| < 0.1$ days 
  (crosses) and $1 < |t_c - t| < 10$ days (circles), for a numerical simulation of 
  the ETAS model. The solid line for each time period, is the predicted magnitude 
  distribution estimated as the sum of the unconditional GR law with an exponent 
  $b = 1.5\beta = 1$, shown as a dashed black line, and a deviatoric GR law $dP(m)$
  with an exponent $b' = b - \alpha = 0.5$ with $\alpha = 1.5a = 0.5$. 
  We must stress that the energy distribution is no more a pure power law close 
  to the mainshock but the {\bf sum of two power laws}.
 %\end{minipage}

 
\end{frame}



\begin{frame}
 {Magnitude rate as the sum of two power laws!}
 
 %\begin{minipage}{0.48\linewidth}
   \includegraphics[width=1\linewidth]{Figs/fig8} \\
 %\end{minipage}
 %\begin{minipage}{0.48\linewidth}
   Same as previous, but for $a = 0.8\beta$. In this case, the deviatoric GR 
   contribution is observed only for the largest magnitudes, for which the 
   statistics is the poorest, hence the relatively large fluctuations around the 
   exact theoretical predictions.
 %\end{minipage}

 
\end{frame}




\begin{frame}
 {Migration of foreshocks as diffusion factor changes}
 
 \begin{minipage}{0.48\linewidth}
   \includegraphics[width=0.8\linewidth]{Figs/fig9}
 \end{minipage}
 \begin{minipage}{0.48\linewidth}
  (opposite) Migration of foreshocks for superposed foreshock sequences
  generated with the ETAS model for two choices of parameters: 
  (a) we see clearly a migration of the seismicity toward the mainshock, 
  larger diffusion exponent H = 0.2. (b) the distribution of the foreshock–mainshock
  distances shown in panel is independent of the time from the mainshock.
  Time periods ranging between $10^{4}$ and $10^{-4}$ days before mainshock.
  The distribution of mainshock–aftershock distances describing direct 
  lineage is shown as a dashed line. The characteristic size
  of the foreshock cluster is shown as a function of the time to
  the mainshock on panel (c) circles correspond to the simulation 
  shown in panel (a) and crosses correspond to the simulation shown in
  panel (b).
 \end{minipage}

\end{frame}




\begin{frame}
 {Seismic rate alarms better than other criteria}
 
% \begin{minipage}{0.48\linewidth}
 \begin{center}
   \vspace{-10pt}
  \includegraphics[width=0.5\linewidth]{Figs/fig10}
 \end{center}
% \end{minipage}
% \begin{minipage}{0.48\linewidth}
   \vspace{-10pt}
   {\footnotesize Results of prediction tests for synthetic catalogs.
   The minimum magnitude is $m_0 = 3$ and the target events are $M > 6$
   (a total of 4735 $M>6$ mainshocks). We use three functions
measured in a sliding window of 100 events: (1) the maximum magnitude M max of the 100 events in
that window, (2) the apparent GR exponent b measured on these 100 events by the standard Hill
maximum likelihood estimator, and (3) the seismicity rate r defined as the inverse of the duration of
the window. For each function, we declare an alarm when the function is either larger (for M max and r)
or smaller (for b) than a threshold. The quality of the predictions is
measured by plotting the ratio of failures to predict as a function of the total durations of the alarms
normalized by the duration of the catalog.}
% \end{minipage}

 
\end{frame}



\end{document}


