
\documentclass[a4paper,10pt]{article} % Default font size and paper size

%\usepackage{fontspec} % For loading fonts
%\defaultfontfeatures{Mapping=tex-text}
%\setmainfont[SmallCapsFont = Fontin SmallCaps]{Fontin} % Main document font
%`  \usepackage{fontspec}
%  \usepackage{polyglossia}
%  \setmainlanguage{french}

\usepackage{url,parskip} % Formatting packages
\setlength{\parskip}{5pt}
\setlength{\parsep}{0pt}
\setlength{\itemsep}{0pt}

\usepackage[usenames,dvipsnames]{xcolor} % Required for specifying custom colors

\usepackage[big]{layaureo} % Margin formatting of the A4 page, an alternative to layaureo can be \usepackage{fullpage}
% To reduce the height of the top margin uncomment: 
%\addtolength{\voffset}{-2cm}
	\addtolength{\topmargin}{-1.1in}
	\addtolength{\textheight}{2.5in}
	\addtolength{\textwidth}{0.7in}

\addtolength{\hoffset}{-1cm}

\usepackage{hyperref} % Required for adding links	and customizing them
\definecolor{linkcolour}{rgb}{0,0.2,0.6} % Link color
\hypersetup{colorlinks,breaklinks,urlcolor=linkcolour,linkcolor=linkcolour} % Set link colors throughout the document

\usepackage{titlesec} % Used to customize the \section command
\titleformat{\section}{\Large\scshape\raggedright}{}{0em}{}[\titlerule] % Text formatting of sections
\titlespacing{\section}{0pt}{3pt}{3pt} % Spacing around sections

\usepackage[dvips,xetex]{graphicx}


\def\logo{%
\begin{picture}(0,0)\unitlength=3cm
\put (0,-0.25) {\includegraphics[width=4.25em]{logo-is.png}}
\end{picture}
}
\def\photo{%
\begin{picture}(0,0)\unitlength=3cm
\put (4.7,-1.1) {\includegraphics[width=6em]{my_orcid_qrcode.png}}
\end{picture}
}
\usepackage{marvosym}


\begin{document}


%\pagestyle{empty} % Removes page numbering

%\font\fb=''[cmr10]'' % Change the font of the \LaTeX command under the skills section

%----------------------------------------------------------------------------------------
%	NAME AND CONTACT INFORMATION
%----------------------------------------------------------------------------------------


%\logo \photo
\photo

\pagestyle{empty} % Removes page numbering

\font\fb=''[cmr10]'' % Change the font of the \LaTeX command under the skills section

%----------------------------------------------------------------------------------------
%	NAME AND CONTACT INFORMATION
%----------------------------------------------------------------------------------------

\par{\centering{\Huge Hugo \textsc{S. S\'anchez Reyes}}\bigskip\par} % Your name


%\section{Personal Data}
\begin{tabular}{rl}
\textsc{Birth:} & 4 July 1989 \\
\textsc{Address:} & 1220 rue de R\'esidences, 38406 Saint Martin d'H\`eres, France \\
\textsc{Contact:} & \Email \, \href{mailto:hugo.sanchez-reyes@univ-grenoble-alpes.fr}{hugo.sanchez-reyes@univ-grenoble-alpes.fr} \quad \Telefon \, +33 07 71 89 30 07\\
& \ComputerMouse \, \href{http://hugosanrocks.github.io/}{http://hugosanrocks.github.io/} \, \ComputerMouse \, \href{https://github.com/hugosanrocks}{https://github.com/hugosanrocks}
\end{tabular}

\section{Work Experience}

\begin{tabular}{rp{11cm}}
{\bf \textsc{Oct. 2015 - Currently }} & {\bf PhD. Student at Universit\'e Grenoble Alpes, ISTerre} \\
& Thesis: \emph{An Adjoint-State Method for 3D Seismic Source Inversion} \\[0.4em] 
%& \footnotesize{Funded by the Centre National de la Recherche Scientifique (CNRS), France}\\
%\multicolumn{2}{c}{} \\
%\end{tabular}

%\begin{tabular}{rp{11cm}}
{\bf \textsc{Mar. 2015 - Jun. 2015}} & {\bf Research Internship at Universit\'e Joseph Fourier, ISTerre} \\
& Project:  ANR S4 \emph{Subduction: standard and slow seismology}\\[0.4em]
%& Tasks: \footnotesize{3D kinematic seismic source inversion of mexican earthquakes.}\\
%\multicolumn{2}{c}{} \\

{\bf \textsc{Feb. 2014 - Jun. 2014}} & {\bf Internship at Universit\'e Joseph Fourier, ISTerre} \\
%& Equipe: Ondes, Stagiaire Master 2 Erasmus Mundus\\ 
& Project: \emph{1D Eikonal equation by a DG Finite Element Method}\\[0.4em]
%\multicolumn{2}{c}{} \\

{\bf \textsc{Oct. 2012 - Aug. 2013}} & {\bf Geophysical Specialist in Exploration at \textsc{COMESA S. A.}} \\
& Tasks: \emph{Development of didactic software for PSTM seismic data.}
%\multicolumn{2}{c}{} \\
\end{tabular}


\section{Education and trainning}

\begin{tabular}{rl}	
{\bf \textsc{Feb.} 2015} & {\bf MSc. in Engineering Seismology} \\
& Thesis: \emph{Eikonal Equation by a DG-FEM} $|$ \small Advisor: Prof. Jean \textsc{Virieux}\\
& \small{Universit\'e Joseph Fourier, France $|$ Istituto Universitario di Studi Superiori di Pavia, Italy}\\[0.4em]
%& \small\emph{Erasmus Mundus} $|$ Major: Engineering Seismology $|$ \textsc{GPA}: 14.109/20.0

{\bf \textsc{Apr.} 2013} & {\bf BSc. in Geophysical Engineering} \\
& Thesis: \emph{Seismic source modeling of Jalisco earthquake (Mw=8.0), 1995, from the} \\
& \emph{inversion of seismic and static records} $|$ \small Advisor: PhD. Vala \textsc{Hjorleifsdottir}\\
& \small{National Autonomous University of Mexico, UNAM, Mexico} $|$ \bf{Graduated with honors} \\[0.4em]
%& Major: Seismology $|$ \textsc{GPA}: 9.18/10.0 $|$ \small\emph{Graduated with honors} \\[0.4em] %{\hfill \\| \footnotesize Degree 5$^{th}$ page}\\

{\bf Other trainings:}\\[0.4em]
%------------------------------------------------
{\bf \textsc{Mar.} 2013} & Guralp-Amp\`ere Latin American Instrumentation Workshop\\
& Basic use and setup of Guralp and FreeWave instruments\\
& \small\textbf{Amp\`ere Instrumentation and Telemetry}, Mexico\\[0.4em]

{\bf \textsc{Jul.} 2011} & Course of field seismic instrumentation\\
& Use of seismic sensors and digitizers\\
& \small\textbf{Institute of Geophysics, National Autonomous University of Mexico}, Mexico\\
\end{tabular}




\section{Scientific publications}
\begin{tabular}{ll}
 [1] & {\bf S\'anchez-Reyes}, H. S., Tago, J., Métivier, L., Cruz-Atienza, V. M. and Virieux, J., 2018. \tabularnewline 
     & An evolutive linear kinematic source inversion. {\it JGR Solid Earth, 123}, DOI: 10.1029/2017JB015388. \href{https://agupubs.onlinelibrary.wiley.com/doi/10.1029/2017JB015388}{[Link]} \\

 [2] & Hjorleifsdottir, V., {\bf S\'anchez-Reyes}, H. S., Ruiz Angulo, A., Ramírez-Herrera, M. T., \tabularnewline
     & Castillo-Aja, R., Singh, S. K. and Ji, C., 2018. Was the October $9^{th}$ 1995 $M_w$ 8 Jalisco, Mexico \\
     & earthquake a near trench event? {\it JGR Solid Earth,} DOI: 10.1029/2017JB014899. \href{https://agupubs.onlinelibrary.wiley.com/doi/10.1029/2017JB014899}{[Link]} \\
     
 [3] & Jara, J., {\bf S\'anchez-Reyes, H. S.}, Socquet, A., Cotton, F., Virieux, J., Maksymowicz et al., 2018. \\
     & Kinematic Study of Iquique 2014 M$_w$8.1 earthquake: understanding the segmentation of \\
     & the seismogenic zone. {\it Earth and Planetary Sciences} DOI: 10.1016/j.epsl.2018.09.025. \href{https://www.sciencedirect.com/science/article/pii/S0012821X18305648}{[Link]}
\end{tabular}

\section{Languages \hspace{0.55cm} \& \hspace{0.55cm} other skills}

\begin{tabular}{{l}{l}|{l}{l}}
\textsc{French:}  & Fluent B2		 & Basic: 		& ArcGIS, SEISAN \\%\setmainfont[SmallCapsFont=Fontin SmallCaps]{Fontin-Regular}\\
\textsc{English:} & Fluent IELTS 7.0 	 & Intermediate: 	& C++, Python, QGIS, OpenQuake, SAC, MPI, Perl, SeismicUnix \\
\textsc{Russian:} & Basic A2		 & Advanced: 		& Fortran, OpenMP, MatLab, GMT, \LaTeX \\  
\textsc{Spanish:} & Mothertongue         & Leadership: 		& 2011-2012 President of UNAM's SEG Student Chapter \\
\end{tabular}



%----------------------------------------------------------------------------------------
%	SCHOLARSHIPS AND ADDITIONAL INFO
%----------------------------------------------------------------------------------------

\section{Scholarships \hspace{2cm} \& \hspace{2cm} Conferences}

\begin{tabular}{{r}{l}|{r}{l}}
2015 & CNRS Recrutement \'etudiant handicap\'e   & 2018 & AGU Fall Meeting ({Oral Participation})  \\
2013 & Erasmus Mundus MEEES 			 & 2017 & IAG-IASPEI Kobe Meeting ({Poster Participation})  \\
2012 & ExxonMobil Research Scholarship  	 & 2016 & AGU Fall Meeting ({Poster Participation}) \\
2011 & Chevron/SEG SLS Travel Grant		 & 2015 & UGM Annual Meeting ({Oral Participation}) \\ 
2018 & Mechanics of Earthquake Faulting	Workshop & 2012 & UGM Annual Meeting ({Oral Participation}) \\
 & & & Posters and presentations: \href{https://www.researchgate.net/profile/Hugo_Sanchez-Reyes}{[GitHub link]}
\end{tabular}

\section{Interests and Activities}

Inverse problems, bayesian inference, programming, languages, saxophone, writing science-fiction

%\begin{tabular}{rl}
%\textsc{Oct.} 2016 & Congress: Poster Participation\\
%& \footnotesize\emph{An evolutive quasi-real-time source inversion based on a linear inverse formulation}\\
%& \footnotesize{2016 Annual Fall Meeting | American Geophysical Union, San Francisco, CA., USA}\\
%&\\
%\textsc{Oct.} 2016 & Congress: Oral Participation\\
%& \footnotesize\emph{3D Seismic source inversion using the Adjoint-state method}\\
%& \footnotesize{2015 Annual Meeting | Mexican Geophysical Union, Puerto Vallarta, Mexico}\\
%&\\
%\textsc{Sept.} 2013 & Scholarship: Erasmus Mundus MEEES Consortium Scholarship \\
%& \footnotesize{Consortium: UJF (France), IUSS (Italy), METU (Turkey), University of Patras (Grece)}\\
%&\\
%\textsc{Jan.} 2012 & Congress: ExxonMobil Research Scholarship\\
%& \footnotesize{\emph{Seismic source modeling of Jalisco earthquake (M$_w$8), 1995, from seismic and static records}}\\
%& \footnotesize{ExxonMobil and Institute of International Education, IIE}\\
%&\\
%\textsc{Sept.} 2011 & Scholarship: Travel Grant, Chevron Student Leadership Symposium \\
%& \footnotesize{Chevron and Society of Exploration Geophysicists}\\
%& \footnotesize{81$^{th}$ SEG Annual Meeting. San Antonio, TX., USA}\\
%&\\
%\textsc{Oct.} 2012 & Congress: Oral Participation\\
%& \footnotesize\emph{Seismic source modeling of Jalisco earthquake (Mw=8.0), 1995, from seismic and static records}\\
%& \footnotesize{2012 Annual Meeting | Mexican Geophysical Union, Puerto Vallarta, Mexico }\\
%\end{tabular}


%\section{Mobility}

%I have decided to study abroad to expand my knowledge, skills and culture on all the fields I am
%interested in. Traveling and living in the UK and France, allowed me to master my second and third
%languages. This skills and cultural exposure encouraged me to pursue a Master Degree in Engineering
%Seismology, which is my main interest, at two of the most recognized European educational institutions
%specialized in this field. In addition, I believe that participating in national and international 
%congresses is the best way to establish working networks with collegues and to be constantly updated 
%on the advances in Seismology which certainly enriches my professional research life.

%As a visually impaired person, I have chosen to develop my professional career in France where
%the well organized and handicap-inclusive society allows me to make the best out of my life.

%----------------------------------------------------------------------------------------
%	COMPUTER SKILLS 
%-----------------------L-----------------------------------------------------------------



%----------------------------------------------------------------------------------------
%	INTERESTS AND ACTIVITIES
%----------------------------------------------------------------------------------------




\end{document}



 [1] & {\bf S\'anchez-Reyes}, H. S., Tago, J., Métivier, L., \tabularnewline
     & Cruz-Atienza, V. M. and Virieux, J. \tabularnewline
     & 2017. An evolutive \\ quasi-real-time \tabularnewline 
     & kinematic source inversion based on a linear \tabularnewline
     & formulation. (Submitted)\\[0.4em]

 [2] & Jara, J., {\bf S\'anchez-Reyes, H. S.}, Socquet, A., Cotton, F., Virieux, J., Maksymowicz, 
       A., Diaz-Mojica, J., \\ Walpersdorf, A., Ruiz, J., Cotte, N. and Norabuena, E., 2017.
       Kinematic Study of Iquique 2014 \\
       Mw 8.1 earthquake: understanding the segmentation of the seismogenic zone. (Submitted)\\[0.4em]
       
 [3] & Hjorleifsdottir, V., {\bf S\'anchez-Reyes}, H. S., Ruiz Angulo, A., Ramírez-Herrera, M. T.,
       Castillo-Aja, R., \\ Singh, S. K. and Ji, C., 2017. Large early afterslip following the
       1995/10/09 M$_w$8 Jalisco, Mexico earthquake.





%----------------------------------------------------------------------------------------
%	WORK EXPERIENCE 
%----------------------------------------------------------------------------------------


%----------------------------------------------------------------------------------------
%	EDUCATION
%----------------------------------------------------------------------------------------



%----------------------------------------------------------------------------------------
%	LANGUAGES
%----------------------------------------------------------------------------------------


%----------------------------------------------------------------------------------------

